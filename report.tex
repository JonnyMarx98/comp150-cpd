% Please do not change the document class
\documentclass{scrartcl}

% Please do not change these packages
\usepackage[hidelinks]{hyperref}
\usepackage[none]{hyphenat}
\usepackage{setspace}
\doublespace

% You may add additional packages here
\usepackage{amsmath}

% Please include a clear, concise, and descriptive title
\title{Continuing Personal Development Report}

% Please do not change the subtitle
\subtitle{COMP150 - CPD Report}

% Please put your student number in the author field
\author{1603748}

\begin{document}

\maketitle

\section{Introduction}

In the future after I complete this course my goal is to become a game developer. I would like to start small and make indie games in a small development team, then If things go well I would love to join a larger company to work on AAA games as a computing professional. This term the five key challenges I faced were version control, essay writing, tinkering, presenting, and game programming. I will highlight these five skills in my report by acknowledging the difficulties i faced then suggesting how i will overcome the obstacle in relation to a SMART action. 

\section{Version Control}

Version control is one of the five challenges i faced. This skill is important to learn and master as it is extremely useful to use when programming with others, especially when programming games. Without proper use of version control many errors will occur causing the program to break. I have used version control on three different assignments this semester, tinkering audio and graphics, as well as the pre-production task. The first time i used it i wasn't using it how it should be used, myself and my pair uploaded separate code and i had to manually merge them together a few days before the deadline. It was not until we were introduced to git branching that I started using version control more appropriately. I found that my biggest difficulty with version control is understanding how to use branching properly. To overcome this obstacle and gain a better understanding of git branching, I will complete the git branching tutorials on learngitbranching.js.org and enter the Git-Game over the holiday to prepare myself for next semesters group game project. This action meets all five of SMART criteria.

\section{Essay Writing and Research}

Essay writing and research is important to master because being able to communicate new influential findings through an essay can be very useful and is it a great skill to have when working in a computing related area. This semester I have written two essays excluding this one which i struggled with a lot. I haven't had any real essay writing or referencing experience before this so i was very new to it. I found the most difficult part was finding relevant information in my research and analysing that information to come up with my own ideas. To overcome this challenge, when given my next essay i will spend a lot more time on it and submit drafts for review so i can see what i have done well and what i could improve on. I make this action measurable and time-bound by setting a goal of submitting at least two drafts for review by the week before the deadline.    

\section{Independent Study}

Independent study is something every student should learn to be good at, especially computing students as we have to also independently practice programming in our own time if we want to be successful. During the months since I started this course I have done a fair amount of independent study, however it has only been when work I need to do for an assignment. I have done hardly any extra programming practice of further reading which could have improved my understanding and knowledge. My most notable difficulty is the amount of time I spend procrastinating rather than working and the fact that I do very little work on assignments that are not due within a few days. I found that I spend more time working on the assignments that I enjoy the most such as pre-production tasks. To overcome these obstacles set myself work goals to complete, such as doing 2 hours of programming practice every week day and rewarding myself if I complete these goals.   

\section{Presenting}

Good presenting is a highly important and relevant skill to have when developing games. This is because you have to present your game ideas to investors and other important people when working in the games industry, such as publishing company's. This semester I have been involved in multiple presentations, only one of which i actually spoke in. This was the individual presentation of our agile questions. My biggest problem with presenting is my confidence in talking in front of an audience and my explaining skills. I am often quiet and stutter when I present to an audience. To overcome this I plan to practice presenting to my friends and family to try and improve my confidence. I will also rehearse presentations alone before the actual presentation to avoid stuttering. I will rehearse everything I want to say until I can say it all without reading anything from the slides, this is something measurable and i will do it the day before presenting to ensure that it is a time-bound action.      

\section{Programming}

Programming is arguably the most important skill to learn and become an expert of for a computing professional in the games industry. This semester there has been many assignments which involve programming. Having learnt the basics of programming in another language before starting this course, learning the basics of python wasn't too challenging. When I started to come across new programming constructs I found it a lot more difficult. For example I still don't fully understand how classes work. In the pre-production task I often used code that my team had written but didn't completely understand how it worked. To overcome this difficulty I will look into every construct that I don't understand and practice using them until I know exactly what it does and how to use it both effectively and appropriately. I will do this over the holiday before the second semester starts to ensure this action is time-bound.    

\section{Conclusion}

In my report I have identified five SMART actions I intend to complete, there are learning to git branch effectively, spending more time on my essays, setting myself work goals, practice presenting, and practicing the programming constructs I am not confident with. These are all actions that will improve my relevant skills to become a good employable computing professional in the games industry. They will also be very helpful for next semesters project, the programming and presentation practice will be especially beneficial to this.   


\bibliographystyle{ieeetran}
\bibliography{references}

\end{document}
